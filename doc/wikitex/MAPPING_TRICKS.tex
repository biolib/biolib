= Mapping tricks =

== Introduction ==

This document documents a number of mapping tricks for the BioLib
development environment. The term 'mapping' is used here for the
build-time creation of Ruby/Perl/Python (etc.) modules against a
shared C library.

== Using CMake ==

CMake is the default choice for BioLib - and, so far, it has worked
raather well. However, CMake has its oddities and one can waste a lot
of time on trying to figure out what happens.

CMake caches data it generates and reuses the cache the next time.
This can lead to confusion when updating CMake configuration files. I
really don't see why it has to be this way, but the best way around it
is to either run the ./scripts/cleanup.sh script, every time, or
slightly more risky:

rm CMakeCache.txt ; cmake .

=== Debugging CMake ===

CMake can be run at any level in the tree. Start from the
smallest unit after running the ./scripts/cleanup.sh script.

The best way to debug CMake is by using MESSAGE statements
breaking, or not, with a FATAL_ERROR attribute. E.g.

  IF(NOT IS_DIRECTORY ${MODULE_SOURCE_PATH})
    MESSAGE(FATAL_ERROR "${MODULE_SOURCE_PATH} does not exist")
  ENDIF(NOT IS_DIRECTORY ${MODULE_SOURCE_PATH})

To print a list:

  # this is used when searching for include files e.g. using the FIND_PATH() command.
  MESSAGE( STATUS "CMAKE_INCLUDE_PATH: " ${CMAKE_INCLUDE_PATH} )

  # this is used when searching for libraries e.g. using the FIND_LIBRARY() command.
  MESSAGE( STATUS "CMAKE_LIBRARY_PATH: " ${CMAKE_LIBRARY_PATH} )

% $

To check whether code should not reach a place:

  MESSAGE(FATAL_ERROR "should not be here!")

=== Building shared libraries ===

In principle cmake can be run at all folder levels. When creating a
mapping - say to Ruby - first make sure the relevant shared libraries
are built correctly. To prevent existing libraries from linking bump
the number in ./VERSION first and run sh ./scripts/cleanup.sh.

Every relevant shared library has to be run in its folder. So:

  cd src/clibs/module-1.0/src
	cmake .
	make

As a convention the shared libraries are built in
./src/clibs/module-1.0/build, so the mapper can find them easily. E.g.

  SET(CMAKE_LIBRARY_OUTPUT_DIRECTORY ../build)

=== Mapping shared libraries ===



=== Include files ===

Include files visible to the mapping process are located by default in
./src/clibs/module-1.0/(src|include). For different locations the
USE_INCLUDE parameter needs to be set.

== Troube shooting ==

=== BIOLIB_LIBRARY-NOTFOUND ===

This means either the BioLib core library has not been built:

  cd ./src/clibs/biolib_core/src
	cmake .
	make
	ls ../build

=== MODULE_LIBRARY=MODULE_LIBRARY-NOTFOUND ===

This means eigher the module library (against which you are mapping) has not
been built:

  cd ./src/clibs/module-version/src
	cmake .
	make
	ls ../build

Or the shared library name does not match the module name. Typically
there will be a mismatch between the names in
./src/clibs/module-version and the build/libmodule-version.so names.

